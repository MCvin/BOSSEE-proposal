\begin{sitedescription}{QS}

\par QuantStack was founded in 2016 by a team of developers and maintainers of key packages of the open-source scientific computing stack. QuantStack provides support and custom development services in the Jupyter and Scientific Python ecosystems. Clients and partners of QuantStack range from financial software companies to robotics startups and public research institutions. The team comprises several core developers of Jupyter subprojects and authors of popular scientific computing and visualization software used in both academic and industrial contexts.

\par Beyond Project Jupyter, projects developed at QuantStack include data visualization packages for Jupyter such as bqplot, ipyvolume, ipyleaflet, and ipysheet, as well as Jupyter language kernels such as xeus-cling and xeus-python, and JupyterLab extensions like te draw.io and sidecar. QuantStack is also behind the development of the xtensor framework, a high-level array computing library and C++ dataframe.

\subsubsection*{Curriculum vitae of the investigators}

\input{CVs/Sylvain.Corlay.tex}
\input{CVs/Johan.Mabille.tex}
\input{CVs/Martin.Renou.tex}
\input{CVs/Wolf.Vollprecht.tex}

% For other to-be-hired person, please include here something like:
\begin{participant}[type=R,PM=41,salary=7000]{Software Engineer} %% Salary: standard SME cost
  We will hire a software engineer with experience working in large open-source
  projects. They will benefit from the mentoring of the other Jupyter contributors
  of the QuantStack team.
\end{participant}

\subsubsection*{Publications, products, achievements}

\begin{compactenum}

\item QuantStack developers participate in the continuous development of \emph{Project Jupyter}. The team is especially active in the area of interactive widgets, as well as JupyterLab and the Jupyter Server.

\item QuantStack is the main driving force behind the \emph{xtensor} project, a C++ tensor expression system for high-performance computing. Xtensor comes along with language bindings for Python, R, and Julia, as well as interfaces to BLAS, FFTW, and means to input and output a large number of standard file formats.

\item QuantStack also develops the \emph{xeus} project, a framework for creating Jupyter language kernels. Xeus is used as a foundation for the C++ Jupyter kernel "xeus-cling", built upon the Cling C++ interpreter from CERN. Xeus was also adopted in Kitware's \emph{Slicer} medical imaging software for its Jupyter integration.

\item The QuantStack team includes the authors and maintainers of some of the most popular Jupyter interactive widgets packages, including \emph{bqplot}, a 2-D interactive plotting system, \emph{ipyvolume}, a 3-D volume rendering package, \emph{ipyleaflet}, a maps visualization toolkit.

\item QuantStack contributes extensively to the \emph{conda-forge} project, a community-maintained collection of packages for scientific computing. Nearly a hundred "recipes" for conda-forge are maintained by QuantStack.

\item QuantStack developers are also behind the \emph{vaex} data decimation engine for interactive visualization of large datasets.

\end{compactenum}

\subsubsection*{Relevant projects or activities}

\par Beyond open-source scientific computing development, QuantStack promotes scientific open source software development through the organization of events and by volunteering in non-profit organizations promoting the ecosystem.

\begin{compactenum}

\item QuantStack team members co-organize the regular \emph{PyData Paris Meetup}, a free event series taking place every two to three months. After a year, the group counts over two thousand members in Paris.

\item We also support the \emph{NumFOCUS Fondation} as volunteers as a member of the team is a member of the board of directors of the foundation.

\end{compactenum}

% \subsubsection*{Significant infrastructure}

\end{sitedescription}

%%% Local Variables:
%%% mode: latex
%%% TeX-master: "../proposal"
%%% End:
