\begin{sitedescription}{XFEL}
  \label{sitedescription:euxfel}

% PIC:
% see: http://ec.europa.eu/research/participants/portal/desktop/en/orga

% See ../proposal.tex, section Members of the Consortium for a
% complete description of what should go there

  European X-Ray Free-Electron Laser Facility GmbH is a limited
  liability company under German law. At present, 12 countries are
  participating in the project: Denmark, France, Germany, Hungary,
  Italy, Poland, Russia, Slovakia, Spain, Sweden, Switzerland, and the
  United Kingdom.  The company is in charge of the operation and
  construction of the European XFEL, a 3.4 km long X-ray free-electron
  laser facility extending from Hamburg to the neighbouring town of
  Schenefeld in the German federal state of Schleswig-Holstein. Civil
  construction started in early 2009, and the user operation in
  September 2017. With its repetition rate of 27,000 pulses per second
  and a peak brilliance a billion times higher than that of the best
  synchrotron X-ray radiation sources, the European XFEL will allow
  the investigation of still open scientific problems in a variety of
  disciplines (physics, structural biology, chemistry, planetary
  science, study of matter under extreme conditions and many others).

  European XFEL has a data policy in place \cite{datapolicy-euxfel}
  which opens up facility data for open access after an embargo period
  of 3 years.
\subsubsection*{Curriculum vitae}

% Curriculum of the personnel at this institution
%
\input{CVs/Hans.Fangohr.tex}
\input{CVs/Sandor.Brockhauser.tex}
\begin{participant}[type=PI,PM=1,gender=male]{Krzysztof Wrona}
  % type is one of:
  % - leadPI: leader of the participating institution
  % - PI: Principal Investigator
  % - R: researcher?
  % Who is the coordinator is specified elsewhere

  % PM=YYY:
  % A fair evaluation of the number of months you will be
  % spending on this specific project along the four years.
  % Typical numbers:
  % - full time hired personnel: 48 months
  % - lead PI or proposal coordinator: 8-12 months
  % - PI: 4-5 months
  % - participant: 2-6 months

  % salary=ZZZ:
  % Approximate monthly gross salary (in term of total cost for the
  % employer). This is optional. If you are uncomfortable having this
  % information in a public file, you can alternatively send the
  % information to Eugenia Shadlova, or to your institution
  % leader/manager if he is willing to fill in himself the budget
  % forms on the eu portal.

  % The above information is used to fill in various tables in the
  % proposal file, and to evaluate the cost of the project for the
  % institutions.

  % You may remove all those comments.

  % About half a page of free text; for whatever it's worth, you may see
  % Nicolas.Thiery.tex for an example.



  \medskip Krzysztof Wrona has a background in computer physics. As
  the group leader of IT and Data Management at European XFEL, he is
  in charge of the management of scientific data in the frame of the
  user program of the European XFEL facility. He has more than 15
  years of experience in data storage, processing, and in general IT
  issues.
\end{participant}

%%% Local Variables:
%%% mode: latex
%%% TeX-master: "../proposal"
%%% End:

\input{CVs/Thomas.Kluyver.tex}
%


\begin{participant}[PM=68, type=R]{Research Engineer x2}

We will hire two postdoctoral-level research software engineers (for 68 person
months in total) to carry out the required work for this projet at
European XFEL. They will work under supervision of Hans Fangohr, with
support from Sandor Brockhauser and Krzysztof Wrona for particular
aspects. The employees will either have a scientific background and
significant software engineering expertise, or an education in
computer science and an aptitude to work with scientists on
computational science and data science problems.
\end{participant}

\subsubsection*{Publications, products, achievements}

\begin{compactenum}
\item H.Fangohr, Python for Computational Science and Engineering
  (2018) DOI: 10.5281/zenodo.1411868 \newline
  https://github.com/fangohr/introduction-to-python-for-computational-science-and-engineering
\item H.Fangohr et al., “Data Analysis support in Karabo at European
  XFEL”, Proceedings of International Conference on Accelerator and
  Large Experimental Physics Control Systems 2017, ISBN 978-3-95450-
  193-9, Data Analytics, Barcelona, Spain, TUCPA01 (2017) DOI: 10.18429/JACoW-ICALEPCS2017-TUCPA01
\item H.Fangohr.
\emph{A Comparison of \software{C}, \Matlab and \Python as Teaching Languages in Engineering}
Lecture Notes on Computational Science \textbf{3039}, 1210-1217 (2004)
\item T. Kluyver, B. Ragan-Kelley, F. Perez, B. Granger, M. Bussonier, J. Frederic, K. Kelley, J. Hamrick, J. Grout, S. Corlay et al. Jupyter
\emph{Notebooks: a publishing format for reproducible computational workflows} In 20th International Conference on Electronic Publishing. IOS Press, 2016.
\end{compactenum}

\subsubsection*{Relevant projects or activities}

\begin{compactenum}
\item OpenDreamKit (GA No. 676541) Open Digital Research Environment
  Toolkit for the Advancement of Mathematics, participant
\item EOSCpilot (GA No. 739563) The European Open Science Cloud for
  Research Pilot Project, participant
\item PaNOSC (GA No. 823852) Photon and Neutron Open Science Cloud,
  participant
\item CALIPSOplus (GA No. 730872) Convenient Access to Light Sources
  Open to Innovation, Science and to the World, participant
\item ATTRACT (GA No. 777222) breAkThrough innovaTion pRogrAmme for a
  pan-European Detection and Imaging eCosysTem, participant

\end{compactenum}




\end{sitedescription}



%KEY-MORE-TODOS



%%% Local Variables:
%%% mode: latex
%%% TeX-master: "../proposal"
%%% End:

%  LocalWords:  sitedescription Programme organisations programmes Centres subsubsection
%  LocalWords:  micromagnetic Nmag Fischbacher Franchin Bordignon Fangohr emph textbf
%  LocalWords:  Multiphysics summarised Iridis TFlops Modelling
