\begin{sitedescription}{INSERM}

Inserm, the French National Institute of Health \& Medical Research is the only
public sector research institution in France exclusively dedicated to human
health. Under the dual aegis of the Ministries of Health and Research, Inserm
has a budget of 900 M euros and employs 15,000 scientists, engineers and
technicians all with one shared objective, namely to promote health - by
advancing knowledge about living organisms and their diseases, developing
innovative treatment modalities and conducting research on public health.

Inserm is represented within the BOSSEE consortium through the Cancer Research
Centre of Toulouse (CRCT) and Inserm’s Computing Department (D\'epartement du
Système D’Information, DSI).

CRCT gathers academic, scientific, medical, clinical, technological and
pharmaceutical research on cancer on a 220-hectares site next to Toulouse,
France. Its missions are to improve fundamental knowledge on all aspects of
cancer biology and to provide patients with rapid access to innovative and
individualized treatments. On these premises, CRCT comprises 21 teams
affiliated to Inserm, the University of Toulouse and the CNRS (National Centre
for Scientific Research). Team 15 of CRCT led by M. Bardi\`es aggregates
Medical Physics resources available in Toulouse around a common research theme:
the optimization of radiotherapy through the development of innovative
dosimetric approaches at various scales (cell, tissue, patient).

Inserm's IT department (DSI) defines and coordinates IT and information systems
aspects across the whole institution. It designs and operates Inserm's
information system to support research activities of the institute, and
provides counselling and support to research units on information technologies.
It also plays a strong role in coordinating IT security and risk management
policies for the French health science community.

% PIC:
% see: http://ec.europa.eu/research/participants/portal/desktop/en/organisations/
%
% See ../proposal.tex, section Members of the Consortium for a
% complete description of what should go there

\subsubsection*{Curriculum vitae}
% Curriculum of the personnel at this institution. This includes
% to-be-hired people for which there is a tentative candidate.
\begin{participant}[type=leadPI,PM=3,gender=male]{Manuel Bardi\`es}
  % type is one of:
  % - leadPI: leader of the participating institution
  % - PI: Principal Investigator
  % - R: researcher?
  % Who is the coordinator is specified elsewhere

  % PM=YYY:
  % A fair evaluation of the number of months you will be
  % spending on this specific project along the four years.
  % Typical numbers:
  % - full time hired personnel: 48 months
  % - lead PI or proposal coordinator: 8-12 months
  % - PI: 4-5 months
  % - participant: 2-6 months

  % salary=ZZZ:
  % Approximate monthly gross salary (in term of total cost for the
  % employer). This is optional. If you are uncomfortable having this
  % information in a public file, you can alternatively send the
  % information to Eugenia Shadlova, or to your institution
  % leader/manager if he is willing to fill in himself the budget
  % forms on the eu portal.

  % The above information is used to fill in various tables in the
  % proposal file, and to evaluate the cost of the project for the
  % institutions.

  % You may remove all those comments.

  % About half a page of free text; for whatever it's worth, you may see
  % Nicolas.Thiery.tex for an example.

  Manuel Bardi\`es, PhD, obtained his doctorate on radiopharmaceutical
  dosimetry from Paul Sabatier University (Toulouse III) in 1991. He has been
  developing his research in radiopharmaceutical dosimetry within INSERM
  (National Institute of Health and Medical Research), since 1992, in Nantes
  then in Toulouse (2011) within the Cancer Research Centre of Toulouse (CRCT).
  He is the responsible of CRCT Team 15 entitled "Multi- resolution dosimetry
  for radiotherapy optimization".
  
  Dr. Bardi\`es has been appointed to several international positions. He was
  one of the founders of the EANM Dosimetry Committee (member from 2001 to
  2013, chair 2009-2011). He also chaired of EFOMP Science Committee
  (2014-2016).
  
  Dr. Bardi\`es is also involved in education and is currently member of the
  Board of the European School for Medical Physics Expert (ESMPE) and member of
  the European School of Multimodality Imaging and Therapy (ESMIT).
  
  The team led by Manuel Bardi\`es in Toulouse (CRCT Team 15) is primarily
  involved in radiopharmaceutical dosimetry, at various scales (cell, tissue,
  organs). This requires the ability to assess radiopharmaceutical
  pharmacokinetics in vivo, through quantitative SPECT or PET small-animal
  imaging. An important part of research activity is related to Monte Carlo
  modelling of radiation transport through biological structures of interest,
  in order to give account of energy deposition within tumour targets - or
  critical non-tumour tissues/organs. The objective is to improve molecular
  radiotherapy by allowing patient-specific treatments, as an important
  application of personalized medicine.

\end{participant}

%%% Local Variables:
%%% mode: latex
%%% TeX-master: "../proposal"
%%% End:

\input{CVs/Maxime.Chauvin.tex}
\begin{participant}[type=PI,PM=6,gender=female]{Isabelle Perseil}
  % type is one of:
  % - leadPI: leader of the participating institution
  % - PI: Principal Investigator
  % - R: researcher?
  % Who is the coordinator is specified elsewhere

  % PM=YYY:
  % A fair evaluation of the number of months you will be
  % spending on this specific project along the four years.
  % Typical numbers:
  % - full time hired personnel: 48 months
  % - lead PI or proposal coordinator: 8-12 months
  % - PI: 4-5 months
  % - participant: 2-6 months

  % salary=ZZZ:
  % Approximate monthly gross salary (in term of total cost for the
  % employer). This is optional. If you are uncomfortable having this
  % information in a public file, you can alternatively send the
  % information to Eugenia Shadlova, or to your institution
  % leader/manager if he is willing to fill in himself the budget
  % forms on the eu portal.

  % The above information is used to fill in various tables in the
  % proposal file, and to evaluate the cost of the project for the
  % institutions.

  % You may remove all those comments.

  % About half a page of free text; for whatever it's worth, you may see
  % Nicolas.Thiery.tex for an example.

  Isabelle Perseil, PhD, is the Head of the Computational Science Coordination
  and e-infrastructures of Inserm. Dr. Perseil manages a group of 3 experts
  which provides the best practices in software engineering, Data Management,
  Big data, deep learning, HPC, Grids, Cloud Computing, parallel computing to
  300 research units (1200 research teams).
  
  The Computational Science Coordination is working with 13 regional
  administrations and 23 regional Mesocenters to pool the computational
  resources (grids and HPC) and train more than 1000 engineers and researchers
  to HPC (OpenMP, MPI and now ORWL) and Big data (MapReduce, Hadoop, Spark,
  Flink, Storm).

\end{participant}

%%% Local Variables:
%%% mode: latex
%%% TeX-master: "../proposal"
%%% End:

\input{CVs/Gilles.Mathieu.tex}

% For other to-be-hired person, please include here something like:
% \begin{participant}[type=res,PM=3,salary=5900]{NN}
%  <a _short_ description of the qualifications of whom you want to hire>
% \end{participant}

\subsubsection*{Publications, products, achievements}
\begin{compactenum}
\item A. Albeyatti et al. “Towards a European health research and innovation
  cloud (HRIC)”. In: 2019 accepted in Genome Medicine.
\item M. Chauvin et al. “OpenDose: Generating reference data for Nuclear
  Medicine dosimetry”. In: European Journal of Nuclear Medicine and Molecular
  Imaging 44.S2 (Sept. 2017), pp. 119--956. DOI: 10.1007/s00259-017-3822-1.
\item D. Salas et al. “Resource-Centered Distributed Processing of Large
  Histopathology Images”. In: 2016 IEEE Intl Conference on Computational
  Science and Engineering (CSE) and IEEE Intl Conference on Embedded and
  Ubiquitous Computing (EUC) and 15th Intl Symposium on Distributed Computing
  and Applications for Business Engineering (DCABES). Aug. 2016, pp. 367--370.
  DOI: 10.1109/CSE-EUC-DCABES.2016.210.
\item S. Marcatili et al. “Model-based versus specific dosimetry in diagnostic
  context: Comparison of three dosimetric approaches”. In: Medical Physics 42.3
  (2015), pp. 1288--1296. DOI: 10.1118/1.4907957.
\item D. Sarrut et al. “A review of the use and potential of the GATE Monte
  Carlo simulation code for radiation therapy and dosimetry applications”. In:
  Medical Physics 41.6 Part1 (2014), p. 064301. DOI: 10.1118/1.4871617.
\item I. Perseil et al. “An Efficient Modeling and Execution Framework for
  Complex Systems Development”. In: 2011 16th IEEE International Conference on
  Engineering of Complex Computer Systems. Apr. 2011, pp. 317--331. DOI:
  10.1109/ICECCS.2011.38.
\item  A. Divoli et al. “Effect of Patient Morphology on Dosimetric
  Calculations for Internal Irradiation as Assessed by Comparisons of Monte
  Carlo Versus Conventional Methodologies”. In: Journal of Nuclear Medicine
  50.2 (2009), pp. 316--323. DOI: 10.2967/jnumed.108.056705.
\item I. Perseil and L. Pautet. “Foundations of a new software engineering
  method for real-time systems”. In: Innovations in Systems and Software
  Engineering 4.3 (Oct. 2008), pp. 195--202. ISSN: 1614-5054. DOI:
  10.1007/s11334-008-0067-y
\end{compactenum}

\subsubsection*{Relevant projects or activities}
Inserm is the leading academic biomedical research institution in Europe with
more than 13,000 publications a year; and second in the world (behind the
American National Institutes of Health).

Inserm has 24 international cooperation agreements, 33 associated European
laboratories (AELs) and associated international laboratories (AILs), and 183
Horizon 2020 contracts since 2014 - of which 45 were signed in 2017. 67 ERC
winners have been hosted at Inserm since 2012, 13 of whom in 2017.  Inserm is
involved in many ESFRIs:
\begin{compactenum}
\item ERINHA2 (H2020)
\item ERINHA (FP7)
\item ADOPT BBMRI-ERIC (H2020)
\item BioMedBridges (FP7)
\item MRTdosimetry EMPIR (H2020)
\item MetroMRT (REG)
\end{compactenum}
Inserm is also one of the funding partners of the French NGI, integrated within
EGI.

\subsubsection*{Significant infrastructure}
Inserm has more than 350 research units spread across France and
internationally. These are supported by 13 Regional Commissions for local
oversight. Scientific activities are organized around 9 “Inserm Thematic
Institutes”, corresponding to the main fields of biomedical and health
research.

\end{sitedescription}
%%% Local Variables:
%%% mode: latex
%%% TeX-master: "../proposal"
%%% End:
