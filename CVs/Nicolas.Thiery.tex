\begin{participant}[type=leadPI,PM=5,gender=male]{Nicolas M. Thiéry}
  Professor at the Laboratoire de Recherche en Informatique, Nicolas
  M. Thiéry is a senior researcher in Algebraic Combinatorics with 18
  papers published in international journals. Among other things, he
  is a member of the permanent committee of FPSAC, the main
  international conference of the domain, a founding member of the
  upcoming Numfocus Europe non-profit, and a member of Work group on
  Free and Open Source software for the ``Open Science Committee'' of
  the French Ministry for Research.
  He has collaborators
  in the US and Canada where he cumulatively spent more than three
  years (Colorado School of Mines, UC Davis, Providence, Montréal),
  and in India. He also
  co-organised fourteen international workshops, in particular \Sage Days, and the semester
  long program on ``Automorphic Forms, Combinatorial Representation Theory and Multiple
  Dirichlet Series'' hosted in Providence (RI, USA) by the Institute for Computational and
  Experimental Research in Mathematics.

  Algebraic combinatorics is a field at the frontier between mathematics and computer
  science, with heavy needs for computer exploration. Pioneer in community-developed open
  source software for research in this field, Thiéry founded in 2000 the \SageCombinat
  software project (incarnated as \MuPADCombinat until 2008); with 50 researchers
  in Europe and abroad, this project has grown under
  his leadership to be one of the largest organised community of Sage developers, gaining
  a leading position in its field, and making a major impact on one hundred
  publications\footnote{\url{http://sagemath.org/library-publications-combinat.html},
    \url{http://sagemath.org/library-publications-mupad.html}}. Along the way,
%this occasion
%Thiéry gained a strong community building experience, and
  he coauthored part of the proposal for NSF \SageCombinat grant
  OCI-1147247, and co-organised or taught at a dozen training and
  dissemination actions (workshops, summer schools, etc.), in
  America, Africa, Europe, and India.

  With 150 tickets (co)authored and as many refereed, Thiéry is himself a core \Sage
  developer, with contributions including key components of the \Sage infrastructure
  (e.g. categories), specialised research libraries (e.g. root systems), thematic
  tutorials, and two chapters of the book ``Calcul Mathématique avec \Sage''
  and its English translation.

  Based on this experience, and to tackle the pressing funding needs
  in the ecosystem of open source mathematical software, Thiéry
  initiated and lead the European Research Infrastructures project
  OpenDreamKit \#676541 (2015-2019, 15 sites, 50 participants, 8M€),
  engaging the Jupyter project on board. This in turn increased his
  involvement in using, promoting, and contributing to Jupyter, for
  use in mathematics and education.
\end{participant}
%%% Local Variables:
%%% mode: latex
%%% TeX-master: "../proposal"
%%% End:
