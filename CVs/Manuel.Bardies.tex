\begin{participant}[type=leadPI,PM=3,gender=male]{Manuel Bardi\`es}
  % type is one of:
  % - leadPI: leader of the participating institution
  % - PI: Principal Investigator
  % - R: researcher?
  % Who is the coordinator is specified elsewhere

  % PM=YYY:
  % A fair evaluation of the number of months you will be
  % spending on this specific project along the four years.
  % Typical numbers:
  % - full time hired personnel: 48 months
  % - lead PI or proposal coordinator: 8-12 months
  % - PI: 4-5 months
  % - participant: 2-6 months

  % salary=ZZZ:
  % Approximate monthly gross salary (in term of total cost for the
  % employer). This is optional. If you are uncomfortable having this
  % information in a public file, you can alternatively send the
  % information to Eugenia Shadlova, or to your institution
  % leader/manager if he is willing to fill in himself the budget
  % forms on the eu portal.

  % The above information is used to fill in various tables in the
  % proposal file, and to evaluate the cost of the project for the
  % institutions.

  % You may remove all those comments.

  % About half a page of free text; for whatever it's worth, you may see
  % Nicolas.Thiery.tex for an example.

  Manuel Bardi\`es, PhD, obtained his doctorate on radiopharmaceutical
  dosimetry from Paul Sabatier University (Toulouse III) in 1991. He has been
  developing his research in radiopharmaceutical dosimetry within INSERM
  (National Institute of Health and Medical Research), since 1992, in Nantes
  then in Toulouse (2011) within the Cancer Research Centre of Toulouse (CRCT).
  He is the responsible of CRCT Team 15 entitled "Multi- resolution dosimetry
  for radiotherapy optimization".
  
  Dr. Bardi\`es has been appointed to several international positions. He was
  one of the founders of the EANM Dosimetry Committee (member from 2001 to
  2013, chair 2009-2011). He also chaired of EFOMP Science Committee
  (2014-2016).
  
  Dr. Bardi\`es is also involved in education and is currently member of the
  Board of the European School for Medical Physics Expert (ESMPE) and member of
  the European School of Multimodality Imaging and Therapy (ESMIT).
  
  The team led by Manuel Bardi\`es in Toulouse (CRCT Team 15) is primarily
  involved in radiopharmaceutical dosimetry, at various scales (cell, tissue,
  organs). This requires the ability to assess radiopharmaceutical
  pharmacokinetics in vivo, through quantitative SPECT or PET small-animal
  imaging. An important part of research activity is related to Monte Carlo
  modelling of radiation transport through biological structures of interest,
  in order to give account of energy deposition within tumour targets - or
  critical non-tumour tissues/organs. The objective is to improve molecular
  radiotherapy by allowing patient-specific treatments, as an important
  application of personalized medicine.

\end{participant}

%%% Local Variables:
%%% mode: latex
%%% TeX-master: "../proposal"
%%% End:
