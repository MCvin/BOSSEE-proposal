\eucommentary{Milestones means control points in the project that help to chart progress. Milestones may
correspond to the completion of a key deliverable, allowing the next phase of the work to begin.
They may also be needed at intermediary points so that, if problems have arisen, corrective
measures can be taken. A milestone may be a critical decision point in the project where, for
example, the consortium must decide which of several technologies to adopt for further
development.}



\paragraph{General Milestones}

\begin{milestones}
  \milestone[
    id=startup,
    month=12,
    verif={
      Completed all corresponding deliverables
      and preparation for deployment of prototype services is underway
      }
    ]
  {Startup, requirements, and prototype generic Jupyter service}
  {
  By milestone 1, we will have established the infrastructure
  for the project and begun prototyping development and deployment of services,
  engaging with the existing communities,
  coordinating plans for \TheProject with those of the wider Jupyter and open science communities,
  and prototyping operation of a generic Jupyter service
  for Open Science.
  EGI Infrastructure-as-a-Service (IaaS) cloud resources are available with a
  Service Level Agreement (SLA) for \TheProject.
  }

  \milestone[
    id=prototype,
    month=24,
    verif={
      Completed all corresponding deliverables and early users are able to access and test prototype services
    }
    ]
  {Generic Jupyter service and early local demonstrators}
  {
  By milestone 2, we will have deployed a generic Jupyter service for Open Science, and begun to experiment with early-adopter
  users and local demonstrators to guide further development of \TheProject,
  ensuring that development serves the needs of the community.
  An initial \TheProject Service Management System shall be available,
  and \TheProject services are integrated with the EOSC Marketplace,
  and AAI Cloud services.
  }

  \milestone[
    id=community,
    month=36,
    verif={Completed all corresponding deliverables and services
    are operational and ready for public testing}
    ]
  {Demonstrator services and community engagement}
  {
  By milestone 3, demonstrator services should be useful and accessible
  to a broad range of users.
  By this point, we will have training materials and run
  workshops to train users in Open Science,
  making use of operational \TheProject services,
  gathering feedback to guide the further development of \TheProject.
  \TheProject Service Management System is fully compliant with the EOSC Service Management practices.
  }

  \milestone[
    id=final,
    month=48,
    verif={Completed all corresponding deliverables and reported progress at the final project review}
  ]
  {Full EOSC integration, adoption, sustainability, and evaluation}
  {
  By this point, all \TheProject services should be operational, TRL 8, and available via EOSC-hub.
  Through community engagement via workshops, conferences, and other media,
  the services should have established groups of users,
  benefiting from these services and improving the Open Science landscape
  on EOSC and beyond.
  At the end of the project,
  we will have engaged with the community to evaluate
  the prototype EOSC services,
  identified which services and tools shall be sustained beyond the life of the project,
  and developed a sustainability plan for
  how this may be achieved under community support and leadership.
  }

\end{milestones}
