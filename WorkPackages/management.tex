\TOWRITE{ALL}{Proofread WP 1 Management pass 1}
\begin{draft}
\TOWRITE{PS (Work Package Lead)}{For WP leaders, please check the following (remove items
once completed)}
\begin{verbatim}
- [ ] have all the tasks in this Work Package a lead institution?
- [ ] have all deliverables in the WP a lead institution?
- [ ] do all tasks list all sites involved in them?
- [ ] does the table of sites and their PM efforts match lists of sites for each task?
      (each site from the table is listed in all relevant tasks, and no site is listed
      only in the table or only at some task)
\end{verbatim}
\end{draft}

\begin{workpackage}[id=management,type=MGT,wphases=0-48!.2,
  title=Project Management,
  short=Management,
  lead=SRL,
  CDSRM=3,
  EGIRM=3,
  EPRM=3,
  INSERMRM=3,
  QSRM=3,
  SILRM=3,
  SRLRM=24,
  UIORM=3,
  UPSUDRM=3,
  WTTRM=3,
  XFELRM=3,
  swsites
]
\begin{wpobjectives}
The main objective of WP1 is to establish and maintain an effective contract, project, and operational management approach ensuring:

 \begin{compactitem}
    \item Timely and successful implementation of the project; including administrative and legal coordination
    \item Technical management and quality assurance
    \item Risk and innovation management of the project as a whole; including data and IPR management
    \item Smooth communication and interaction with the EC and other interested parties

 \end{compactitem}
\end{wpobjectives}

\begin{wpdescription}
The project will be managed by Simula, which has extensive experience in administering and leading EU funded and national projects. The coordinator together with the WP leaders, will be responsible for monitoring WP status, coordination of work plan updates and annual internal progress reports. The project management structure and roles of partners in the consortium are presented in \ref{sect:mgt}.

\end{wpdescription}

\begin{tasklist}

\begin{task}[
  title=Administrative Management,
  id=admin,
  lead=SRL,
  PM=24,
  wphases={0-48},
  partners={CDS,EGI,EP,INSERM,QS,UIO,UPSUD,SIL,WTT,XFEL}
]
The task includes the following activities:
\begin{compactenum}
\item Preparation, distribution and maintenance of all contractual documents (Consortium Agreement, Grant Agreement and all other legal frameworks)
\item Establishment of appropriate communication and collaborative environment for the consortium, as well as the EC and other relevant academic and industry stakeholders (the project website, intranet and communication procedures) to organise transfer of knowledge, present and promote project results (\localdelivref{infrastructure});
\item Organisation of project review and progress meetings;
\item Performing qualitative and quantitative risk analysis, planning risk mitigation and control
\item Progress and Financial Reporting to the EC;
\item Data and IPR Management will be managed in accordance with agreed rules stated in the Consortium Agreement and in accordance with the Data Management Plans (\localdelivref{data-management-plan}, \localdelivref{innovation-management-plan}).
\end{compactenum}
\end{task}

\begin{task}[
  title=Technical Project Management,
  id=project-management,
  lead=SRL,
  PM=24,
  wphases={0-48},
  partners={CDS,EGI,EP,INSERM,QS,UIO,UPSUD,SIL,WTT,XFEL}
]
The project scientific and technical management ensures coherent quality and soundness of the work and results. A quality assurance plan will be developed by UPS, involving all partners, and will be followed up regularly. It will address the reviews and approval of technical reports and deliverables. In addition, the Project Coordinator with the help of the coordination team will regularly review technological risks and recommend mitigation plans to minimise or remove them. This will be reported on at each Reporting Period in the project's Technical Report.
\end{task}

\begin{task}[
  title=Innovation Management,
  id=innovation-management,
  lead=SRL,
  PM=6,
  wphases={0-48},
  partners={CDS,EGI,EP,INSERM,QS,UIO,UPSUD,SIL,WTT,XFEL}
]
One of the most important criteria for success for the BOSSEE project is to bring the project results into use. Therefore, exploitation routes will be sought whenever possible. In order to create a common understanding within the Consortium of how we can best shepherd an idea all the way from conception to realisation and exploitation, the Coordinator will be responsible for the preparation and realisation of an Innovation Plan. This plan will assure that research activities meet the required milestones and produce outputs fully aligned with the project objectives. All research activities will go through an initial process where the exploitation opportunity is identified along with the main stakeholders for the exploitation opportunity and an IP owner
(\localdelivref{innovation-management-plan}).
\end{task}

\end{tasklist}


\begin{wpdelivs}

\begin{wpdeliv}[due=1,miles=startup,id=infrastructure,dissem=PU,nature=DEC,lead=SRL]
  {Basic project infrastructure (websites, wikis, issue trackers, mailing lists, repositories)}
\end{wpdeliv}

\begin{wpdeliv}[due=9,miles=startup,id=data-management-plan-draft,dissem=PU,nature=R,lead=SRL]
  {Data Management Plan draft}
\end{wpdeliv}

\begin{wpdeliv}[due=9,miles=startup,id=innovation-management-plan,dissem=CO,nature=R,lead=SRL]
  {Innovation Management Plan}
\end{wpdeliv}

\begin{wpdeliv}[due=48,miles=final,id=data-management-plan,dissem=PU,nature=R,lead=SRL]
  {Data Management Plan}
\end{wpdeliv}

\end{wpdelivs}
\end{workpackage}
%%% Local Variables:
%%% mode: latex
%%% TeX-master: "../proposal"
%%% End:

%  LocalWords:  workpackage wphases wpobjectives wpdescription pageref wpdelivs wpdeliv
%  LocalWords:  dissem mailinglists swrepository final-mgt-rep compactitem swsites ipr
%  LocalWords:  TOWRITE tasklist delivref
