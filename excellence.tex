% \subsection{Context and motivation}

In many scientific disciplines, it is common for researchers to rely on
heterogeneous computational tools and technologies to collect data,
explore the input data sets, run simulations, visualise the outcome,
and share their result with peers or a with a larger audience. Often,
such data analysis cycles are iteratively refined.

For simple datasets, processes may remain manageable. However, when
dealing with larger and more complex use cases, including big data
from research facilities or High Performance Computing resources, the
complexity makes iteration cycles slower for the researchers. A
complex iteration cycle also makes research results more difficult to
reproduce.
Results that cannot be reproduced make research ineffective: they
create barriers towards re-using the results in future research work,
a critical aspect of Open Science.
This situation is exacerbated by the current and accelerating increase of the amount
of scientific data being available, including the data becoming
accessible through the EOSC-Hub. But this growing availability of data also provides a massive opportunity
for Open Science.

Project Jupyter has developed as one piece of various solutions to the data deluge,
by enabling the construction of computational services accessible from
any where, any device,
with access to any data. Jupyter-based tools such as \href{http://mybinder.org}{Binder} and repo2docker
show great promise for enabling researchers to better perform \textbf{Reproducible and Open Science}.
Jupyter was recognised for its contribution to data analysis in research with the prestigious 2017 \emph{ACM Software System Award}, of which previous winners include TCP/IP, UNIX, and the World Wide Web.
It is widely used today in research,
education, and industry.
We will build on these tools,
both improving their capabilities
and expanding their accessibility to new communities,
both academic and demographic,
in order to \textbf{further the mission of Open Science}.

In this proposal, core team members of Jupyter projects -- including a
number of recipients of the \emph{ACM Software System Award} -- and key contributors to the
open source scientific computing ecosystem,
detail improvements to the
capabilities of Project Jupyter
to \textbf{provide a framework on which innovative EOSC services can be created}.
By collaborating with a wide variety of stakeholders from diverse
scienctific and educational domains,
we aim to demonstrate and ensure that such
innovative EOSC services -- built on Project Jupyter -- are feasible, valuable, and effective in furthering Open Science.
The goal is to \textbf{improve the
accessibility of EOSC resources to researchers and the general public,
and improve the accessibility, interactivity, reproducibility, and
re-usability of computational research
and Open Science.}


%%HF: the following seemed to be to specialised to list in the opening
%%pararaphs?
%
%which
%is especially harmful in scientific software engineering where most innovation
%is achieved through \emph{incrementalism}.




\clearpage

%%% Local Variables:
%%% mode: latex
%%% TeX-master: "proposal"
%%% End:
