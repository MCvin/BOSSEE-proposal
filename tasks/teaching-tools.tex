\begin{task}[
  title={Teaching tools, infrastructure, and best practices},
  id=teaching-tools,
  lead=EP,
  PM=21, % EP: 19PM, UPSud: 2PM
  wphases={0-36!.7},
  partners={UPSUD}
  ]

  This task is devoted to improving the Jupyter ecosystem for
  education. See page \pageref{sec:concept-demonstrator-teaching} for
  context and a list of other tasks that will contribute to better
  teaching.

  Setting up a comfortable working environment for both teachers and
  students requires tools for easy sharing, collecting, self
  assessment, and semi-automatic grading of course material, class
  management, and integration with the local e-learning infrastructure
  such as, e.g., Moodle or OpenEDX.

  We will \textbf{review the state of the art}: existing tools,
  within the Jupyter ecosystem (e.g. nbgrader \cite{Hamrick2016} or OK\cite{OKpy}) and outside;
  course services (e.g. Gryd\cite{Gryd} or CoCalc\cite{colcalc}); course infrastructure that
  have been designed and deployed at many institutions (Berkeley,
  École Polytechnique, Université Paris Sud), etc.

  To build and share a better vision of the needs, we will \textbf{conduct a
  survey in the education community about the usage of Jupyter}. In
  particular we will seek feedback from Jupyter-based MOOCs (Massively Open Online Courses),
  e.g. on
  Coursera\footnote{\url{https://www.coursera.org/courses?query=jupyter}},
  and
  Fun\footnote{\url{https://www.fun-mooc.fr/courses/course-v1:inria+41016+session01bis/about}}.

  The collected requirements will be exposed in a first report
  \delivref{ecosystem}{teaching-report} and largely disseminated.

  The core of the task will then be to \textbf{further develop
    teaching tools, infrastructure, and course templates to contribute
    to the emergence of versatile solutions and best practices around
    them}.

  The outcome will be put into production by the participants of the
  BOSSEE project (and beyond!) who will deliver a large number of
  courses using Jupyter technology (see \taskref{applications}{teaching}). The variety of
  use cases and infrastructure will provide a rich test bed and
  immediate feedback at each iteration, ensuring that the developments
  are informed and steered by demand (co-design), and battle field
  tested.

  At this stage, we already envision specific development in the
  following directions:
  \begin{compactitem}
  % \item Review and follow up on related efforts: gryd.us, cocalc, Coursera/Fun,
  %   Berkeley, Ecole polytechnique, University of Paris Sud, ...
  % \item Survey of the needs in the education community
  \item Collaborative grade management
  \item Insulation through container of the automatic grading
  \item Integration with e-learning platforms (e.g. Moodle, OpenEDX
    (Coursera/Fun)), through an LTI connector.
  \item Develop course templates for various use cases.
  \item Disseminate tutorials on all of the above.
  \end{compactitem}

  A second report \delivref{ecosystem}{nbgrader-like} will review the
  developments, our in-class experience with them, and best practices.
\end{task}
