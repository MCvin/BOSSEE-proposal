\begin{task}[
  title=Demonstrator: enriched teaching with Jupyter,
  id=teaching,
  lead=EP,
  PM=5, % EP: 3PM, UPSUD: 2PM, EuXFEL 1 PM
  wphases={1-48},
  partners={EGI,INSERM,UIO,UPSUD,XFEL}
  ]


  In this task (see page \pageref{sec:concept-demonstrator-teaching}
  for context), we will \textbf{deliver and help deliver Jupyter-based
    courses at a large scale} in our own institutions, as a mean to
  \textbf{inform, evaluate, provide feedback on, and demonstrate the
    value of the work performed} in this project in the context of
  higher education, as well as to \textbf{develop and share best
    practices} and \textbf{demonstrate and disseminate} Jupyter's full
  potential for teaching.

  \'Ecole polytechnique and Université Paris-Sud are particularly well
  suited for this task because they
  \begin{enumerate}
  \item host a variety of local infrastructure (dedicated servers,
    local cloud, computer labs, ...);
  \item host a reactive community with highly qualified research
    software engineers (DevOps, software developers), researchers,
    professors, and students that have been working together on this
    topic for several years, with close collaboration between the two
    sites;
  \item offer very diverse courses, in many disciplines, and ranging
    from large lower undergraduate courses to specialized classes for
    graduate students and top notch engineers;
  \item have strong support from their respective teaching departments.
  % \item have an influential alumni structure through which the
  %   technology will be propagated.\TODO{do we want this item?}
  % \item will create a reactive community of students and researchers,
  %   where the dissemination of the project tools and experience will
  %   be easy, which will organize Jupyter days as in
  %   2018\footnote{\url{http://www.cmap.polytechnique.fr/~massot/Personal_web_page_of_Marc_Massot/JupyterX.html}}
  %   on a regular basis.
  \end{enumerate}

  % \TODO{HF: Loic, can you complete this, please?}

  % A variety of courses are delivered at Université Paris Sud using
  % Jupyter technologies. This includes for example programming classes
  % in C++ at lower undergraduate level (400 students per year since
  % 2017), a series of undergraduate and graduate math courses (computer
  % aided mathematics, computer algebra, numerical methods), or courses
  % in physics, bio-informatics, etc. To support these courses, a
  % JupyterHub service has been deployed in 2017 and progressively
  % improved since, on Paris Sud's local cloud infrastructure, enabling
  % students and teachers to work from anywhere and any device.

  % The Mathematics department at Ecole polytechnique has started a reform of their various teaching offer based on
  % Jupyter for two years and several courses of the Bachelor program, 2nd and 3rd
  % year of Engineering school and Master program have already begun relying on a
  % strong use of Jupyter notebooks / JupyterHub\footnote{MAP551 - 2nd yer course
  % MAP411 - AMS X02 - Mooc INRIA preciser?} and this will continue with a strong
  % support of the Dean of undergraduate studies and of graduate studies. Besides
  % several software and research engineers in applied mathematics have been
  % recruited and participate in this effort, as well as a administration engineer
  % in order to help in terms of building an infrastructure dedicated to Jupyter
  % with the support of the head of the Ecole polytechnique in order to disseminate
  % the effort into various other departments (Physics, Mechanical Engineering,
  % Biology...), where already some courses are starting based in Jupyter. The
  % link with the computer science club of students of Ecole polytechnique (Binet
  % R\'eseau) has also been created with the project and a community is emerging.

  The task will include the following activities
  \begin{compactitem}
  \item Reinforce the use of Jupyter technology in courses at
    all levels, notably in Mathematics and Data Science, in close
    collaboration between Ecole polytechnique and Université Paris Sud;
  \item Test the new developments and feed back to tasks
    \taskref{core}{jh-bh-conv}, \taskref{ecosystem}{xeus-cpp}, \taskref{ecosystem}{teaching-tools}, \taskref{applications}{math} and \taskref{eosc}{jh-bh-deployment};
  \item Follow up on a successful Jupyter day in
    2018~\footnote{\url{http://www.cmap.polytechnique.fr/~massot/Personal_web_page_of_Marc_Massot/JupyterX.html}}
    by organizing a yearly Jupyter event showcasing the latest
    advances for teaching and research;
  \item Foster sharing of experience, best practices and course
    material, at the local level, and then worldwide, through meetups,
    blogs, etc.
  \item Publish selected teaching material for interactive use on
    \TheProject's EOSC services (\delivref{applications}{demonstrators}).
  \end{compactitem}
  The work carried out will be reported on in
  \delivref{applications}{applications-report}.
\end{task}
