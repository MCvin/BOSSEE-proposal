\begin{task}[
  title=Further development of repo2docker and Binder,
  id=r2d-and-binder,
  lead=SRL,
  PM=36,
  wphases={0-48!.75},
  partners={WTT,XFEL}
]
  Running someone else's analyses is a particularly difficult problem.
  There are differences between operating systems, versions of installed software and access to the required data sets.
  These challenges mean that is currently considered to be beyond the scope of an expert peer reviewer to verify data science analysis codes before publication.
  BinderHub, part of Project Jupyter, enables one-click running of git repositories.
  BinderHub provides a web interface to the repo2docker tool.

  The task includes the following activities
  \begin{compactitem}
  \item extend repo2docker with support for execution on cloud resources
  \item extend repo2docker with support for execution on HPC resources with Docker support
  \item improved "first use" experience of running repo2docker locally
  \item add support for using archives such as Zenodo as source for repo2docker and BinderHub
  \item define procedures and recommendations for long term reproducibility and sustainability of repo2docker compatible repositories
  \item create educational material describing repo2docker and its benefits to researchers
  \item Enable Openshift based deployments of BinderHub
  \item User surveys about pain points using BinderHub
  \item User authentication in BinderHub
  \end{compactitem}
\end{task}
