\begin{task}[
  title=Demonstrator: Astronomy,
  id=astro,
  lead=CDS,
  PM=18,
  wphases={18-42},
  partners={EGI,INSERM,QS,SRL,WTT,XFEL}
]

  This task (see page \pageref{sec:concept-demonstrator-astronomy} for
  context) will build on existing Python libraries to access CDS data
  \TODO{Recall what the acronym CDS stands for?}
  (\textit{astroquery.[cds/simbad/vizier/xmatch]}). For visualization, we will
  use proven tools like \textit{GLUE} and \textit{ipyvolume}, which are now
  built upon the Jupyter stack.
  We will also make significant improvement to existing Jupyter widgets
  (\textit{ipyaladin}, interactive sky atlas running in the notebook) and
  develop a new widget to offer a tree-like view of available
  datasets.\TODO{will this be specific to certain needs in astronomy or general purpose?}

  We will also develop Python libraries to allow integration and usage in
  notebook of existing CDS infrastructure services, namely CDSLogin (which
  provides authentication) and CDS MyData (remote storage space for tabular
  data).
  This will allow the user to interact with one's personal storage space from
  the notebook. It will also allow for advanced customisation of the interface
  to fit user needs.

  The work is organised with a two stage approach. Firstly, the generated
  notebooks will run locally on user machines (representing a milestone for
  this task). Following the Binder development in \WPref{eosc}, we will aim
  to run these notebooks on the European Binder Service. The aspiration is
  that this contributes to the development of innovative services for the EOSC.
  The deliverable of this task will be a demonstrator available to the
  scientific user community (\localdelivref{demonstrators}).

  By milestone 3, astronomical data services based on reference
  astronomy data from CDS will be made available in Jupyter notebooks, and
  decision point on how to use developments of WP5 for running these
  notebooks on \TheProject's EOSC services.

  The work carried out in this task will be reported on in
  \delivref{applications}{applications-report}.
\end{task}
