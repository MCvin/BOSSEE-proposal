%\TO WRITE{ALL}{Proofread 3.4 consortium pass 2 [Done by Hans]}
%remove this, as we have more pressing things left.

\eucommentary{\begin{compactitem}
\item
Describe the consortium. How will it match the project's objectives?
How do the members complement one another (and cover the value chain,
where appropriate)? In what way does each of them contribute to the
project? How will they be able to work effectively together?
\item
If applicable, describe the industrial/commercial involvement in the
project to ensure exploitation of the results and explain why this is
consistent with and will help to achieve the specific measures which
are proposed for exploitation of the results of the project (see section 2.3).
\item
Other countries: If one or more of the participants requesting EU funding
is based in a country that is not automatically eligible for such funding
(entities from Member States of the EU, from Associated Countries and
from one of the countries in the exhaustive list included in General
Annex A of the work programme are automatically eligible for EU funding),
 explain why the participation of the entity in question is essential to
 carrying out the project
\end{compactitem}
}

The \TheProject consortium spans the broad spectrum of actors required
for successfully developing an apt and easy-to-navigate sustainable service
accessible through the EOSC hub catering to the needs of the European
scientific community. It is composed of four academic institutions, four research
centers, one e-Infrastructure federation, and two SMEs based in six different countries (Norway, France,
Netherlands, Germany, Poland, Switzerland).
The Consortium ensures a critical mass of scientific expertise and excellence
in key areas (astronomy, geosciences, health sciences,
mathematics, photon science, education) with research organisations and SMEs of recognised
 international reputation. Namely, \TheProject consortium brings in:
\begin{compactitem}
\item A set of use cases that cover several application domains and users, and that impose very diverse
requirements on EOSC infrastructure (European XFEL, CNRS-ObAS, UiO, INSERM, Paris Sud);
\item Lead developers in the Jupyter Ecosystem, including IPython, the Jupyter Notebook, JupyterLab,
JupyterHub, Binder, MyBinder.org, Jupyter Widgets located at Simula, European XFEL, QuantStack, and
WildTreeTech,
as referenced in section \ref{jupyter-ecosystem}.
\item Experts and major promoters of the Jupyter collaborative user interfaces for interactive and exploratory
computing in a variety of scientific domains (Centre de Donnees
astronomiques de Strasbourg, European XFEL,
INSERM, QuantStack, Paris-Sud, \'Ecole Polytechnique, Silesia,
University of Oslo).
\item A long experience and proven track record of success with large and complex collaborative projects,
including
European E-Infrastructure projects (XFEL, Simula, UPSud, Silesia),
projects focused on large-scale infrastructures and large experimental services (EGI, XFEL),
as well as experience in running large scale open source projects (Jupyter project).
\item A comprehensive range of skill sets and competencies in several relevant domains, from applied
research to standardisation to business
analysis.
\end{compactitem}

The consortium has developed through collaborations and common interests over recent years.
Some partners have been working together on different aspects of Jupyter
and software for education for many years (European XFEL, QuantStack, Simula, Wild Tree Tech).
Meanwhile, others joined together during a previous successful
H2020 European Research Infrastructure project OpenDreamKit \#676541 (European XFEL,
Paris-Sud, Silesia, Simula).
OpenDreamKit's review meetings and participation to H2020
E-infrastructure events, as well as community Jupyter workshops (organized by
e.g. by \'Ecole Polytechnique, Paris-Sud, Simula, UiO) led to
connections with UiO, INSERM, and EGI, and to collaborations such as a
prototype deployment of Binder on EGI's infrastructure. The connection
with CNRS-ObAS grew out of a workshop coorganized back in 2013 by
Paris-Sud on mathematical databases, where CDS's head Françoise Genova
was invited speaker; this later led to her joining OpenDreamKit's
Advisory Board.

Finally, we note that the project partners are long time passionate
advocates of Open Science;
building on highly successful past experience with OpenDreamKit, they
\emph{have chosen to write this proposal fully in the open} on GitHub
(\href{http://github.com/bossee-project/proposal}{http://github.com/bossee-project/proposal}) for maximum transparency
and engagement of the community.
We have used the same open source collaboration tools and practices
as the Open Source Open Science community.

In addition to the joint planning and writing period for this
proposal, the project partners have been interacting through a number of
other activities, including:

\begin{enumerate}
\item Joint software development
  \begin{itemize}
  \item Jupyter Notebook (\site{XFEL}, \site{SRL}, \site{WTT}, \site{QS})
  \item Jupyter Widgets (\site{QS}, \site{SIL}, \site{UPSUD}, \site{SRL})
  \item Binder and repo2docker (\site{SRL}, \site{WTT})
  \item thebelab (\site{QS}, \site{SRL}, \site{UPSUD})
  \item nbval (\site{XFEL}, \site{SRL})
  \end{itemize}

\item Joint projects
  \begin{itemize}
  \item \href{http://opendreamkit.org}{OpenDreamKit} (\site{XFEL}, \site{SRL}, \site{UPSUD}, \site{SIL})
  \item PaNOSC (\site{EGI}, \site{XFEL})
  \item Computing in high school science education - iCSE4school, Erasmus+ Strategic Partnerships,
  (\site{SRL}, \site{SIL}), 2014-2017
  \item Computers in Science Education:iCSE (\site{SIL}, \site{UIO}), funded by EFS, 2011-2014.
    \end{itemize}

\item Joint publications
  \begin{itemize}
  \item \emph{Jupyter Notebooks -- a publishing format for
      reproducible computational workflows} \cite{Kluyver2016} (\site{XFEL}, \site{SRL}, \site{QS})
  \end{itemize}

\item Collaboration
  \begin{itemize}
  \item MyBinder for teaching and reproducibility (\site{XFEL}, \site{SRL}, \site{WTT})
  \item Widgets for computational science (\site{XFEL}, \site{QS}, \site{SIL})
  \item OpenGATE, Monte Carlo platform for medical applications (
    \site{INSERM}, \site{UPSUD})
  \item Life Science Grid Community (\site{EGI}, \site{INSERM})
  \item JupyterDays events organised at \'Ecole Polytechnique, Paris-Sud,
  and other sites, with participation from other sites
  (\site{CDS}, \site{EP}, \site{UPSUD}, \site{SRL}).
  \item Research Bazaar workshops on Jupyter, Binder, and reproducibility
    (\site{SRL}, \site{UIO})
 \end{itemize}
\end{enumerate}

Table \ref{tab:collaboration} shows a summary of the links
between partners.

\TOWRITE{ALL}{Add previous collaborations}

% joint software/database development
% Jupyter Project software

\jointsoft{XFEL,SRL}
\jointsoft{WTT,SRL}
\jointsoft{WTT,XFEL}

% Binder
\jointsoft{SRL,WTT}

% k3d
\jointsoft{SIL,SRL}
\jointsoft{SIL,XFEL}
\jointsoft{SRL,XFEL}

% nbval
\jointsoft{XFEL,SRL}

%%s

% OpenDreamKit: UPSUD, SIL, XFEL, SRL
\jointproj{XFEL,UPSUD}
\jointproj{XFEL,SRL}
\jointproj{XFEL,SIL}

\jointproj{UPSUD,SRL}
\jointproj{UPSUD,SIL}

\jointproj{SRL,SIL}
\jointproj{UIO,SIL}

% jupyterhub deployment
\jointproj{EP,UPSUD}

% Binder persistent storage
\jointproj{UPSUD,WTT}
\jointproj{EP,WTT}

% xeus-cling
\jointsoft{QS,EP}
\jointsoft{QS,UPSUD}

% prototype Binder deployment on EGI
\jointproj{Simula, EGI}

% panosc
\jointproj{XFEL,EGI}

% Jupyter project publication ? XXX TIM

% Binder
\jointsoft{SRL,WTT}

% research bazaar
\jointproj{SRL,UIO}

% JupyterDays Orsay + ecole
\jointproj{UPSUD,EP}
\jointproj{UPSUD,CDS}
\jointproj{UPSUD,SRL}
\jointproj{UPSUD,QS}

\jointproj{EP,CDS}
\jointproj{EP,SRL}
\jointproj{EP,QS}

% \jointproj{CDS,SRL}
% \jointproj{CDS,QS}

% \jointproj{SRL,QS}

% OpenGATE
\jointproj{INSERM,UPSUD}

% Life Sciences Grid
\jointproj{INSERM,EGI}

% \jointpub{A,B} % some publication

%joint supervision
% \jointsup{A,B} %

%joint organization
% \jointorga{A,B} % some org
% \jointorga{SA,UJF} % PASCO'15

% joint publications
% \jointpub{A,B} % some publication

% Jupyter publication
\jointpub{SRL,XFEL}
\jointpub{SRL,QS}
\jointpub{XFEL,QS}

\coherencetable[swsites]

%%% Local Variables:
%%% mode: latex
%%% TeX-master: "proposal"
%%% End:

%%% Local Variables:
%%% mode: latex
%%% TeX-master: "proposal"
%%% End:
