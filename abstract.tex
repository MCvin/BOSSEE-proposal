\TODO{In this abstract we want to summarize the vision of the project;
  - Context: Open Science; what is Jupyter; Jupyter is big
  - Who are we
  - What is our goal
  - What is our strategy / concept
Points to hit:
- Open Science should be practical, not just available
- Jupyter is part of the solution
- *brief* highlight of how/why Jupyter and Binder make sense:
  - Jupyter is widely adopted
  - notebook encapsulate computation
  - Binder builds on Jupyter to enable shareable reproducible environments
  - Jupyter is web-based, enabling building services
- What we plan to do
  - improve Jupyter/Binder toward open science
  - operate Jupyter-based services on EOSC
  - Open Science training (skip?)
- Who we are
  - Core Jupyter experts
  - Domain experts motivating/validating Jupyter improvements
}


% \begin{verbatim}
% Call: Prototyping Innovative Services for European Open Science Cloud

% Title: Building Open Science Services for European E-infrastructure (BOSSEE)

% Points to hit:

% - Open Science should be practical, not just available
% - Jupyter is part of the solution
% - *brief* highlight of how/why Jupyter and Binder make sense:
%   - Jupyter is widely adopted
%   - notebook encapsulate computation
%   - Binder builds on Jupyter to enable shareable reproducible environments
%   - Jupyter is web-based, enabling building services
% - What we plan to do
%   - improve Jupyter/Binder toward open science
%   - operate Jupyter-based services on EOSC
%   - Open Science training (skip?)
% - Who we are
%   - Core Jupyter experts
%   - Domain experts motivating/validating Jupyter improvements
% \end{verbatim}

\begin{abstract}

  To truly achieve the societal goals of Open Science,
  we must make progress beyond the `mere availability' of scientific results,
  to the practical usability and exploitation of such data once it is made available,
  an area where there is much room for improvement.
  The Jupyter project and its ecosystem show great promise
  as tools for bridging this gap; for making Open Science
  useful and accessible to all,
  from researchers to educators to public citizens.
  The Jupyter notebook and Jupyter ecosystem are of increasing
  importance in computational science, data science, academia,
  industry, governments, and service providers,
  used by millions worldwide.
  Jupyter notebooks have great potential to push Open Science
  forward because they provide a complete description of a
  computational study that can be turned into a publication
  or produce part of a publication, such as a figure,
  making complex tasks reproducible.
  The Jupyter-based Binder project adds a means to execute notebooks
  in specified computational environments, an aspect of reproducibility
  not yet widely supported
  and a great opportunity to improve Open Science practices.
  The web-based nature of Jupyter makes it an ideal component for Open Science Services,
  being accessible with only a web browser
  while executing anywhere from a local laptop to a remote supercomputer.
  In \TheProject, we will extend the capabilities of the Jupyter
  tools and ecosystem to add functionality that we view as having great
  importance for EOSC and Open Science more
  widely and operate services on EOSC as a demonstration.

  Many \TheProject partners have longstanding experience and
  leadership roles in the Jupyter ecosystem,
  and in deploying services built on Jupyter to many users across the globe.
  Complementary to this core expertise,
  we integrate partners focussing on the application of these tools from a wide range of disciplines,
  both to demonstrate and ensure that our developments serve real-world Open Science use cases.
\end{abstract}

%%% Local Variables:
%%% mode: latex
%%% TeX-master: "proposal"
%%% End:
